%==============================================================================
% Sjabloon poster bachproef
%==============================================================================
% Gebaseerd op document class `a0poster' door Gerlinde Kettl en Matthias Weiser
% Aangepast voor gebruik aan HOGENT door Jens Buysse en Bert Van Vreckem

\documentclass[a0,portrait]{hogent-poster}

% Info over de opleiding
\course{Graduaatsproef}
\studyprogramme{Graduaat in het Programmeren}
\academicyear{2024-2025}
\institution{Hogeschool Gent, Valentin Vaerwyckweg 1, 9000 Gent}

% Info over de bachelorproef
\title{E-commerce platform with Java Spring Boot and React}

\author{Zülal Toy}
\email{zulal.toy@student.hogent.be}
\supervisor{Luc Vervoort}


% Indien ingevuld, wordt deze informatie toegevoegd aan het einde van de
% abstract. Zet in commentaar als je dit niet wilt.
\specialisation{Programmeren}
\keywords{E-commerce, JWT Authenticatie, Stripe API, React, Spring Boot, Full-Stack Webontwikkeling}
\projectrepo{https://github.com/zulaltoy/Zulal-Toy-GP}

\begin{document}

\maketitle

\begin{abstract}
Deze graduaatsproef focust op het bouwen van een veilig en modern e-commerceplatform met Java Spring Boot en React. De applicatie bevat gebruikersauthenticatie met JWT, sessiebeheer via refresh tokens en een geïntegreerde Stripe-betaalfunctie. Het doel is een proof-of-concept te ontwikkelen die laat zien hoe kleine tot middelgrote bedrijven een veilige en gebruiksvriendelijke webshop kunnen opzetten zonder afhankelijk te zijn van bestaande platformen zoals Shopify. Deze oplossing kan dienen als voorbeeld voor andere ontwikkelaars.
\end{abstract}

\begin{multicols}{2} % This is how many columns your poster will be broken into, a portrait poster is generally split into 2 columns

\section{Introductie}

In het digitale tijdperk van vandaag zijn veilige en gebruikersvriendelijke e-commerceplatformen essentieel geworden. Veel kleine tot middelgrote ondernemingen willen hun producten online verkopen,maar willen daarbij niet afhankelijk zijn van grote commerciële  systemen zoals Shopify of WooCommerce.

Deze graduaatsproef richt zich op de ontwikkeling van een modern e-commerceplatform dat veilige gebruikersauthenticatie, sessiebeheer en betalingsfunctionaliteit combineert. Hiervoor worden moderne webtechnologieën zoals Java Spring Boot, React en Stripe gebruikt.

Een belangrijke aspect is het implementeren van JWT-authenticatie en refresh tokens, zodat gebruikers veilig kunnen inloggen en hun sessie behouden blijf. Daarnaast wordt Stripe geïntegreerd voor betalingen ,zodat klanten eenvoudig en veilig online kunnen afrekenen.

De centrale onderzoeksvraag is:
Hoe kan een veilig en gebruikersvriendelijk e-commerceplatform worden opgebouwd met moderne technologieën zoals Spring Boot en React?

\section{Wat heb ik geleerd}

Tijdens het ontwikkelen van dit e-commerceproject heb ik veel bijgeleerd over zowel frontend als backend ontwikkenling. Enkele belangrijke inzichten en vaardigheden die ik heb opgedaan zijn:


\begin{itemize}
\item \textbf{JWT authenticatie:} Ik heb hoe je veilige login en sessiebeheer implementeert met access en refresh tokens.
\item \textbf{Stripe integratie:} Ik weet nu hoe je online betalingen op veilige manier integreert via Stripe API.
\item \textbf{Gebruikersbeheer met rollen:} Ik begrijp nu hoe je toegang kunt beperken tot bepaalde onderdelen van een applicatie op basis van gebruikersrollen.
\item \textbf{Volledige full-stack workflow:} Ik heb ervaring opgedaan met het bouwen, testen, en verbinden van front-end en back-end via REST API’s.
\item \textbf{Probleemoplossend denken:} Tijdens het project ben ik verschillende technische problemen tegenkomen die ik zelfstandig heb kunnen oplossen door te zoeken en te experimenteren.
\end{itemize}
 Door dit project voel ik mij zekerder in het bouwen webapplicaties en heb ik meer vertrouwen gekregen in mijn programmeervaardigheden.

\section{Sectie met figuur}


\begin{center}
  \captionsetup{type=figure}
  \includegraphics[width=1.0\linewidth]{graphics/ui-cozy.eps}
  \captionof{figure}{Home pagina van Cozycollections web pagina.}
\end{center}
\begin{center}
  \captionsetup{type=figure}
  \includegraphics[width=1.0\linewidth]{graphics/cart.eps}
  \captionof{figure}{Winkelmaandje weergave.}
\end{center}



\section{Conclusies}

Deze graduaatsproef toont aan dat het mogelijk is om met moderne technologieën zoals Spring Boot, React en Stripe een veilig en gebruiksvriendelijk e-commerceplatform te bouwen. Door het correct toepassen van JWT-authenticatie, sessiebeheer met refresh tokens en geïntegreerde betalingen, kunnen kleine bedrijven een professionele webshop opzetten zonder afhankelijk te zijn van bestaande CMS-platformen.

\section{Toekomstig onderzoek}

In toekomstig onderzoek kan gekeken worden naar de integratie van andere betalingsproviders zoals PayPal, ondersteuning voor meertaligheid en de uitbreiding met een mobiele applicatie. Ook kan de beveiliging verder worden verbeterd door bijvoorbeeld two-factor authentication toe te voegen of rate limiting op API-endpoints te implementeren. 

\end{multicols}
\end{document}
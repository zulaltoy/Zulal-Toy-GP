%%=============================================================================
%% Samenvatting
%%=============================================================================

% TODO: De "abstract" of samenvatting is een kernachtige (~ 1 blz. voor een
% thesis) synthese van het document.
%
% Een goede abstract biedt een kernachtig antwoord op volgende vragen:
%
% 1. Waarover gaat de graduaatsproef?
% 2. Waarom heb je er over geschreven?
% 3. Hoe heb je het onderzoek uitgevoerd?
% 4. Wat waren de resultaten? Wat blijkt uit je onderzoek?
% 5. Wat betekenen je resultaten? Wat is de relevantie voor het werkveld?
%
% Daarom bestaat een abstract uit volgende componenten:
%
% - inleiding + kaderen thema
% - probleemstelling
% - (centrale) onderzoeksvraag
% - onderzoeksdoelstelling
% - methodologie
% - resultaten (beperk tot de belangrijkste, relevant voor de onderzoeksvraag)
% - conclusies, aanbevelingen, beperkingen
%
% LET OP! Een samenvatting is GEEN voorwoord!

%%---------- Nederlandse samenvatting -----------------------------------------
%
% TODO: Als je je graduaatsproef in het Engels schrijft, moet je eerst een
% Nederlandse samenvatting invoegen. Haal daarvoor onderstaande code uit
% commentaar.
% Wie zijn/haar graduaatsproef in het Nederlands schrijft, kan dit negeren, de inhoud
% wordt niet in het document ingevoegd.

\IfLanguageName{english}{%
\selectlanguage{dutch}
\chapter*{Samenvatting}
\lipsum[1-4]
\selectlanguage{english}
}{}

%%---------- Samenvatting -----------------------------------------------------
% De samenvatting in de hoofdtaal van het document

\chapter*{\IfLanguageName{dutch}{Samenvatting}{Abstract}}


Deze graduaatsproef gaat over het ontwikkelen van een modern e-commerceplatform.

Het onderwerp werd gekozen vanuit mijn persoonlijke interesse in webontwikkeling en online winkels. Ik wilde begrijpen hoe een webshop technisch functioneert, in het bijzonder hoe gebruikers producten aan een winkelmandje kunnen toevoegen en hoe betalingen op de achtergrond worden verwerkt.

De centrale onderzoeksvraag luidde: "Hoe kan een veilige en gebruiksvriendelijke e-commerceplatform worden opgebouwd met moderne webtechnologieën zoals Java Spring Boot en React?"

Het doel van deze graduaatsproef was om een functioneel platform te bouwen waarin gebruikers producten kunnen bekijken, toevoegen aan een winkelmandje, bestellingen plaatsen en betalingen uitvoeren via Stripe. Daarnaast bevat het platform ook een beheerderspaneel voor het beheren van producten en bestellingen.

De backend is ontwikkeld met Java Spring Boot, terwijl de frontend is gebouwd met React en Redux Toolkit. Voor de layout werd uiteindelijk gekozen voor Bootstrap in plaats van Tailwind CSS, om de styling eenvoudiger te houden. Voor het onderzoek heb ik gebruikgemaakt van officiële documentatie, online cursussen (zoals op Udemy) en educatieve YouTube-video’s.

Het resultaat is een volledig werkend platform met basisfunctionaliteiten zoals productbeheer, winkelwagen, orderbeheer en betalingen via Stripe. 

Tijdens het project ben ik tegen verschillende uitdagingen aangelopen. Zo was het integreren van het Stripe-betalingssysteem complex en tijdrovend. Ook het aanleren van een nieuwe programmeertaal (Java), het begrijpen van het Spring Boot-framework en het bouwen van een frontend met Redux Toolkit vergde veel inspanning. Daarnaast zorgde de combinatie van backend, frontend en betalingssysteem ervoor dat de omvang van het project groter was dan oorspronkelijk ingeschat.

In de toekomst kan dit platform verder uitgebreid worden met extra functies zoals gebruikersrecensies, voorraadbeheer per locatie en meertalige ondersteuning.

Deze graduaatsproef toont aan dat het bouwen van een moderne webshop uitdagend is, maar haalbaar met de juiste leerstrategie. Een belangrijk inzicht is dat een goed afgebakend projectdoel en tijdsplanning essentieel zijn om complexiteit beheersbaar te houden.

\chapter{GitHub repository}

De volledige code en documentatie van het project zijn beschikbaar op GitHub:\\\url{https://github.com/zulaltoy/Zulal-Toy-GP}

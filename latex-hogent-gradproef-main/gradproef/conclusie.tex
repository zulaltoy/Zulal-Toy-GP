%%=============================================================================
%% Conclusie
%%=============================================================================

\chapter{Conclusie}%
\label{ch:conclusie}

% TODO: Trek een duidelijke conclusie, in de vorm van een antwoord op de
% onderzoeksvra(a)g(en). Wat was jouw bijdrage aan het onderzoeksdomein en
% hoe biedt dit meerwaarde aan het vakgebied/doelgroep? 
% Reflecteer kritisch over het resultaat. In Engelse teksten wordt deze sectie
% ``Discussion'' genoemd. Had je deze uitkomst verwacht? Zijn er zaken die nog
% niet duidelijk zijn?
% Heeft het onderzoek geleid tot nieuwe vragen die uitnodigen tot verder 
%onderzoek?

De onderzoeksvraag van dit project was: "Hoe kan een veilige en functionele e-commerce platform ontwikkeld worden met Java Spring Boot en React?" Tijdens dit project heb ik geleerd hoe ik een full-stack applicatie kan opbouwen, met aandacht voor gebruikersauthenticatie, productbeheer, winkelmandfunctionaliteit en betalingen via Stripe.

Mijn bijdrage aan het vakgebied is dat ik een werkend prototype heb gebouwd dat als basis kan dienen voor verdere ontwikkeling. Voor beginnende ontwikkelaars biedt mijn project een praktische kijk op hoe een moderne e-commerce applicatie gestructureerd wordt.

Ik had verwacht dat het project uitdagend zou zijn, maar het was moeilijker dan voorzien. Java Spring Boot bleek complexer dan ik dacht en de integratie van alle componenten (zoals Stripe en JWT) vergde veel tijd en opzoekingswerk. Dankzij documentatie, YouTube, Udemy en hulp van ChatGPT heb ik stap voor stap oplossingen gevonden.

Toch zijn er nog zaken die ik niet volledig kon afronden of optimaliseren, zoals sommige validaties in de frontend en uitgebreide errorhandling. Een mogelijke vervolgonderzoeksvraag zou kunnen zijn: "Hoe implementeer je schaalbare beveiliging en foutafhandeling in een e-commerce platform?"

Dit project heeft mijn technische vaardigheden sterk verbeterd en mij veel geleerd over zelfstandig leren, probleemoplossing en softwareontwikkeling in het algemeen.


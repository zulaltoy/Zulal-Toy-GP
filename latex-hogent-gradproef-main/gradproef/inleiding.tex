%%=============================================================================
%% Inleiding
%%=============================================================================

\chapter{\IfLanguageName{dutch}{Inleiding}{Introduction}}%
\label{ch:inleiding}
De COVID-19-pandemie heeft de digitalisering in vele sectoren versneld. Vooral in de retailsector werd het belang van betrouwbare en veilige online betaalmethoden duidelijk. Kleine en middelgrote ondernemingen zagen zich genoodzaakt om digitale oplossingen te implementeren om hun klanten beter van dienst te kunnen zijn. In dit kader ontwikkelde ik een proof-of-concept van een e-commerceplatform met gebruikersauthenticatie, betalingsverwerking en productbeheer, gebruikmakend van Java Spring Boot en React.js.

Het thema van dit onderzoek is het bouwen van een veilig en gebruiksvriendelijk e-commerceplatform met integratie van moderne technologieën. De doelgroep zijn kleine handelaars die op zoek zijn naar een betaalbare oplossing om hun producten online te verkopen.

Veel van deze kleine bedrijven beschikken niet over de technische kennis of middelen om zelf een platform te ontwikkelen. Bestaande oplossingen zijn vaak te duur of niet flexibel genoeg. De centrale onderzoeksvraag is dan ook: \textbf{Hoe kan een veilige en functionele e-commerceoplossing worden ontwikkeld voor kleine handelaars, gebruikmakend van Java Spring Boot en React, met integratie van een moderne betaalmethode zoals Stripe?}

Het doel van dit onderzoek is om een werkend prototype te bouwen dat voldoet aan de belangrijkste behoeften van kleine online verkopers: gebruikersauthenticatie (via JWT), productbeheer, en veilige online betalingen (via Stripe). Dit prototype zal dienen als bewijs dat dergelijke technologieën toegankelijk en toepasbaar zijn in een praktische casus.



\section{\IfLanguageName{dutch}{Probleemstelling}{Problem Statement}}%
\label{sec:probleemstelling}
Veel kleine bedrijven willen hun producten online verkopen, maar hebben niet genoeg technische kennis om een veilige en functionele webshop te laten ontwikkelen. Bestaande platformen zijn vaak duur of te complex. Er is dus nood aan een eenvoudig, uitbreidbaar en betaalbaar e-commerceplatform op maat van kleinere handelaars. Deze graduaatsproef richt zich op het ontwikkelen van zo’n oplossing.



\section{\IfLanguageName{dutch}{Onderzoeksvraag}{Research question}}%
\label{sec:onderzoeksvraag}
Hoe kan een veilige en functionele e-commerceoplossing worden ontwikkeld voor kleine handelaars, gebruikmakend van Java Spring Boot en React, met integratie van een moderne betaalmethode zoals Stripe?



\section{\IfLanguageName{dutch}{Onderzoeksdoelstelling}{Research objective}}%
\label{sec:onderzoeksdoelstelling}

Het doel van deze bachelorproef is het ontwikkelen van een proof-of-concept van een e-commerceplatform dat de volgende functies ondersteunt:

Gebruikersregistratie en -authenticatie via JWT

Productbeheer

Verwerking van online betalingen via Stripe

Het eindresultaat moet een werkend prototype zijn dat als basis kan dienen voor verdere ontwikkeling of integratie in bestaande systemen.



\section{\IfLanguageName{dutch}{Opzet van deze graduaatsproef}{Structure of this associate thesis}}%
\label{sec:opzet-graduaatsproef}

% Het is gebruikelijk aan het einde van de inleiding een overzicht te
% geven van de opbouw van de rest van de tekst. Deze sectie bevat al een aanzet
% die je kan aanvullen/aanpassen in functie van je eigen tekst.

De rest van deze graduaatsproef is als volgt opgebouwd:

In Hoofdstuk~\ref{ch:stand-van-zaken}  wordt de stand van zaken besproken, inclusief relevante technologieën zoals Java Spring Boot, React, JWT en Stripe.

In Hoofdstuk~\ref{ch:methodologie}  wordt de methodologie uitgelegd: hoe het project werd aangepakt, welke tools gebruikt werden, en hoe het ontwikkelingsproces verliep.

% TODO: Vul hier aan voor je eigen hoofstukken, één of twee zinnen per hoofdstuk



In Hoofdstuk~\ref{ch:conclusie}, volgt een conclusie en aanbevelingen voor toekomstig onderzoek of uitbreiding van het platform.
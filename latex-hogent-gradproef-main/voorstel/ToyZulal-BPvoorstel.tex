%==============================================================================
% Sjabloon onderzoeksvoorstel bachproef
%==============================================================================
% Gebaseerd op document class `hogent-article'
% zie <https://github.com/HoGentTIN/latex-hogent-article>

% Voor een voorstel in het Engels: voeg de documentclass-optie [english] toe.
% Let op: kan enkel na toestemming van de graduaatsproefcoördinator!

\documentclass{hogent-article}

% Invoegen bibliografiebestand
\addbibresource{voorstel.bib}

% Informatie over de opleiding, het vak en soort opdracht
\studyprogramme{Graduaat in het Programmeren}
\course{Graduaatsproef}
\assignmenttype{Onderzoeksvoorstel}
% Voor een voorstel in het Engels, haal de volgende 3 regels uit commentaar
% \studyprogramme{Associate Degree in Programming}
% \course{Associate thesis}
% \assignmenttype{Research proposal}

\academicyear{2024-2025} % TODO: pas het academiejaar aan

% TODO: Werktitel
\title{E-commerce Applicatie met Java Spring Boot en React}

% TODO: Studentnaam en emailadres invullen
\author{Zülal Toy}
\email{zulal.toy@student.hogent.be}

% TODO: Medestudent
% Gaat het om een graduaatsproef in samenwerking met een student in een andere
% opleiding? Geef dan de naam en emailadres hier
% \author{Yasmine Alaoui (naam opleiding)}
% \email{yasmine.alaoui@student.hogent.be}

% TODO: Geef de co-promotor op
%\supervisor[Co-promotor]{S. Beekman (Synalco, \href{mailto:sigrid.beekman@synalco.be}\
%{sigrid.beekman@synalco.be})}

% Binnen welke specialisatierichting uit 3TI situeert dit onderzoek zich?
% Kies uit deze lijst:
%
% - Mobile \& Enterprise development
% - AI \& Data Engineering
% - Functional \& Business Analysis
% - System \& Network Administrator
% - Mainframe Expert
% - Als het onderzoek niet past binnen een van deze domeinen specifieer je deze
%   zelf
%
\specialisation{Mobile \& Enterprise development}
\keywords{Java, Spring Boot, React, E-commerce, Web Development}

\begin{document}

\begin{abstract}
  Dit onderzoeksvoorstel beschrijft de ontwikkeling van een full-stack e-commerce applicatie met Java Spring Boot en React. De applicatie biedt gebruikersauthenticatie, productbeheer, een winkelwagentje, betalingen en een beheerderspaneel. Het onderzoek richt zich op het toepassen van moderne webtechnologieën en best practices in softwareontwikkeling. 
\end{abstract}

\tableofcontents

% De hoofdtekst van het voorstel zit in een apart bestand, zodat het makkelijk
% kan opgenomen worden in de bijlagen van de graduaatsproef zelf.
%---------- Inleiding ---------------------------------------------------------



\section{Inleiding}%
\label{sec:inleiding}
In de digitale economie van vandaag zijn e-commerceplatformen onmisbaar geworden voor zowel kleine als grote bedrijven. Klanten verwachten een veilige, betrouwbare en gebruiksvriendelijke online winkelervaring. Een belangrijke uitdaging daarbij is het correct beheren van gebruikersauthenticatie, sessies en betalingsverkeer — drie kritieke componenten die essentieel zijn voor een professioneel e-commerceplatform.

Deze graduaatsproef kadert binnen de ontwikkeling van een modern en veilig e-commerceplatform waarbij gefocust wordt op drie technologieën: authenticatie via JSON Web Tokens (JWT), sessiebeheer met refresh tokens, en betalingintegratie via Stripe. Het project is specifiek gericht op kleine tot middelgrote e-commercebedrijven die hun platform willen moderniseren met veilige en efficiënte technologieën, zonder daarbij afhankelijk te zijn van bestaande platformen zoals Shopify of WooCommerce.

De concrete probleemstelling is dat veel beginnende e-commerceplatformen een zwakke of onveilige implementatie hebben van gebruikersauthenticatie en betalingssystemen. Vaak worden tokens onveilig opgeslagen of wordt de betaalflow niet goed geïntegreerd met de backend, wat leidt tot veiligheidsrisico's of gebruikersproblemen. De centrale onderzoeksvraag luidt dan ook:


\begin{quote}
Hoe kan een veilige en gebruiksvriendelijke e-commerceplatform worden opgebouwd met moderne webtechnologieën zoals Java Spring Boot en React?
\end{quote}

De onderzoeksdoelstelling is om een proof-of-concept (PoC) te ontwikkelen van een veilige en functionele e-commerce webapplicatie waarbij de hierboven genoemde technologieën correct en volgens best practices worden toegepast. Dit wordt ondersteund door een literatuurstudie en technische analyse van bestaande oplossingen. Het einddoel is een prototype dat inzetbaar is in een realistische context en dat andere ontwikkelaars kunnen gebruiken als referentie of basis voor verdere ontwikkeling.

De meerwaarde van dit onderzoek zit in het feit dat er veel tutorials en documentatie bestaan over elk afzonderlijk onderdeel (zoals JWT of Stripe), maar weinig bronnen die deze technologieën op een veilige en coherente manier combineren in één project. Door deze leemte te vullen, biedt dit project een concrete en toepasbare oplossing voor een veelvoorkomend probleem in de e-commercesector.

% Waarover zal je graduaatsproef gaan? Introduceer het thema en zorg dat volgende zaken zeker duidelijk aanwezig zijn:

% \begin{itemize}
%   \item kaderen thema
%   \item de doelgroep
%   \item de probleemstelling en (centrale) onderzoeksvraag
%   \item de onderzoeksdoelstelling
% \end{itemize}

% Denk er aan: een typische graduaatsproef is \textit{toegepast onderzoek}, wat betekent dat je start vanuit een concrete probleemsituatie in bedrijfscontext, een \textbf{casus}. Het is belangrijk om je onderwerp goed af te bakenen: je gaat voor die \textit{ene specifieke probleemsituatie} op zoek naar een goede oplossing, op basis van de huidige kennis in het vakgebied.

% De doelgroep moet ook concreet en duidelijk zijn, dus geen algemene of vaag gedefinieerde groepen zoals \emph{bedrijven}, \emph{developers}, \emph{Vlamingen}, enz. Je richt je in elk geval op it-professionals, een bachelorproef is geen populariserende tekst. Eén specifiek bedrijf (die te maken hebben met een concrete probleemsituatie) is dus beter dan \emph{bedrijven} in het algemeen.

% Formuleer duidelijk de onderzoeksvraag! De begeleiders lezen nog steeds te veel voorstellen waarin we geen onderzoeksvraag terugvinden.

% Schrijf ook iets over de doelstelling. Wat zie je als het concrete eindresultaat van je onderzoek, naast de uitgeschreven scriptie? Is het een proof-of-concept, een rapport met aanbevelingen, \ldots Met welk eindresultaat kan je je bachelorproef als een succes beschouwen?



%---------- Stand van zaken ---------------------------------------------------

\section{Literatuurstudie}%
\label{sec:literatuurstudie}


Bij het ontwikkelen van moderne webapplicaties is gebruikersauthenticatie een van de belangrijkste onderdelen. JSON Web Tokens (JWT) worden vaak gebruikt voor stateless authenticatie in RESTful APIs~\autocite{Jones2015}. JWT laat toe om gebruikersinformatie op een veilige manier te versleutelen en door te sturen in de HTTP-header.

Refresh tokens worden meestal gebruikt om een sessie langdurig actief te houden zonder dat de gebruiker zich telkens opnieuw moet aanmelden. Dit zorgt voor een betere gebruikerservaring, maar vereist een veilige implementatie, bijvoorbeeld door refresh tokens in een httpOnly cookie op te slaan en server-side sessiebeheer toe te passen~\autocite{OWASP2023}.

Voor betalingen is Stripe een van de populairste en meest gedocumenteerde oplossingen. Stripe biedt een krachtige API en ondersteunt token-based betalingen die veilig zijn voor gevoelige klantgegevens~\autocite{StripeAPI}. De integratie met back-end vereist echter zorgvuldige validatie van betalingstokens en foutafhandeling.

Deze technologieën – JWT, refresh tokens en Stripe – zijn allemaal voorbeelden van moderne webtechnologieën die bijdragen aan een veilige en gebruiksvriendelijke e-commerceplatform. Ze worden in dit project gecombineerd met frameworks zoals Spring Boot en React.

% Hier beschrijf je de \emph{state-of-the-art} rondom je gekozen onderzoeksdomein, d.w.z.\ een inleidende, doorlopende tekst over het onderzoeksdomein van je graduaatsproef. Je steunt daarbij heel sterk op de professionele \emph{vakliteratuur}, en niet zozeer op populariserende teksten voor een breed publiek. Wat is de huidige stand van zaken in dit domein, en wat zijn nog eventuele open vragen (die misschien de aanleiding waren tot je onderzoeksvraag!)?

% Je mag de titel van deze sectie ook aanpassen (literatuurstudie, stand van zaken, enz.). Zijn er al gelijkaardige onderzoeken gevoerd? Wat concluderen ze? Wat is het verschil met jouw onderzoek?

% Verwijs bij elke introductie van een term of bewering over het domein naar de vakliteratuur, bijvoorbeeld~\autocite{Hykes2013}! Denk zeker goed na welke werken je refereert en waarom.

% Draag zorg voor correcte literatuurverwijzingen! Een bronvermelding hoort thuis \emph{binnen} de zin waar je je op die bron baseert, dus niet er buiten! Maak meteen een verwijzing als je gebruik maakt van een bron. Doe dit dus \emph{niet} aan het einde van een lange paragraaf. Baseer nooit teveel aansluitende tekst op eenzelfde bron.

% Als je informatie over bronnen verzamelt in JabRef, zorg er dan voor dat alle nodige info aanwezig is om de bron terug te vinden (zoals uitvoerig besproken in de lessen Research Methods).

% Voor literatuurverwijzingen zijn er twee belangrijke commando's:
% \autocite{KEY} => (Auteur, jaartal) Gebruik dit als de naam van de auteur
%   geen onderdeel is van de zin.
% \textcite{KEY} => Auteur (jaartal)  Gebruik dit als de auteursnaam wel een
%   functie heeft in de zin (bv. ``Uit onderzoek door Doll & Hill (1954) bleek
%   ...'')

% Je mag deze sectie nog verder onderverdelen in subsecties als dit de structuur van de tekst kan verduidelijken.

%---------- Methodologie ------------------------------------------------------
\section{Methodologie}%
\label{sec:methodologie}

Deze graduaatsproef is een technische proof-of-concept (PoC) waarbij verschillende technologieën worden gecombineerd in één webapplicatie. De methodologie bestaat uit vier grote fasen:

\begin{enumerate}
  \item \textbf{Analyse en literatuuronderzoek:} Eerst worden bestaande implementaties en best practices onderzocht voor JWT-authenticatie, refresh token gebruik, Stripe-integratie en frontendontwikkeling met React.
  \item \textbf{Ontwikkeling van het backend:} De backend wordt gebouwd met Java Spring Boot. Authenticatie en sessiebeheer worden geïmplementeerd via JWT en refresh tokens. Er wordt gewerkt met gebruikersrollen (admin, gebruiker).
  \item \textbf{Ontwikkeling van de frontend:} De frontend wordt gebouwd met React en Redux Toolkit. Functionaliteiten zoals productweergave, winkelmandje, gebruikersinteractie en orderbeheer worden via een gebruiksvriendelijke interface aangeboden.
  \item \textbf{Integratie van Stripe:} Stripe wordt geïntegreerd als betaalsysteem. Dit vereist backendlogica voor het aanmaken van betaalverzoeken en bevestiging van succesvolle betalingen, evenals frontendcomponenten voor gebruikersinput en betaling.
\end{enumerate}

Voor het ontwikkelen wordt gebruik gemaakt van volgende tools:

\begin{itemize}
  \item Java, Spring Boot, Spring Security
  \item React, Redux Toolkit, Javascript
  \item MySQL als databank
  \item Stripe Developer Dashboard + REST API
  \item Postman voor testen van endpoints
  \item Git en GitHub voor versiebeheer
\end{itemize}

Een ruwe tijdsinschatting:

\begin{itemize}
  \item Week 1-2: literatuuronderzoek en ontwerp
  \item Week 3-5: JWT + refresh token implementatie (backend)
  \item Week 6-7: frontendontwikkeling met React + Redux
  \item Week 8: Stripe-integratie, testen, documenteren en afwerken
\end{itemize}

Elke fase resulteert in een functioneel onderdeel van de applicatie, dat apart getest wordt.
% Hier beschrijf je hoe je van plan bent het onderzoek te voeren. Welke onderzoekstechniek ga je toepassen om elk van je onderzoeksvragen te beantwoorden? Gebruik je hiervoor literatuurstudie, interviews met belanghebbenden (bv.~voor requirements-analyse), experimenten, simulaties, vergelijkende studie, risico-analyse, PoC, \ldots?

% Valt je onderwerp onder één van de typische soorten graduaatsproeven die besproken zijn in de lessen Research Methods (bv.\ vergelijkende studie of risico-analyse)? Zorg er dan ook voor dat we duidelijk de verschillende stappen terug vinden die we verwachten in dit soort onderzoek!

% Vermijd onderzoekstechnieken die geen objectieve, meetbare resultaten kunnen opleveren. Enquêtes, bijvoorbeeld, zijn voor een graduaatsproef informatica meestal \textbf{niet geschikt}. De antwoorden zijn eerder meningen dan feiten en in de praktijk blijkt het ook bijzonder moeilijk om voldoende respondenten te vinden. Studenten die een enquête willen voeren, hebben meestal ook geen goede definitie van de populatie, waardoor ook niet kan aangetoond worden dat eventuele resultaten representatief zijn.

% Uit dit onderdeel moet duidelijk naar voor komen dat je graduaatsproef ook technisch voldoen\-de diepgang zal bevatten. Het zou niet kloppen als een graduaatsproef informatica ook door bv.\ een student marketing zou kunnen uitgevoerd worden.

% Je beschrijft ook al welke tools (hardware, software, diensten, \ldots) je denkt hiervoor te gebruiken of te ontwikkelen.

% Probeer ook een tijdschatting te maken. Hoe lang zal je met elke fase van je onderzoek bezig zijn en wat zijn de concrete \emph{deliverables} in elke fase?

%---------- Verwachte resultaten ----------------------------------------------
\section{Verwacht resultaat, conclusie}%
\label{sec:verwachte_resultaten}
Het verwachte resultaat is een werkende proof-of-concept van een veilige e-commercebackend met gebruikersauthenticatie, sessiebeheer en betalingsfunctionaliteit. Dit omvat:

\begin{itemize}
  \item Een backend API met endpoints voor registratie, login, tokenvernieuwing en gebruikersbeheer.
  \item Een veilige refresh token-flow met httpOnly cookies.
  \item Een geïntegreerd Stripe-betaalsysteem voor het verwerken van bestellingen.
\end{itemize}

Deze toepassing moet aantonen dat het mogelijk is om op een veilige en gebruiksvriendelijke manier een modern e-commerceplatform te bouwen zonder afhankelijk te zijn van commerciële CMS'en. De doelgroep — ontwikkelaars of bedrijven die een eigen platform willen opzetten — kunnen deze PoC gebruiken als uitgangspunt.

Een bijkomend resultaat is een technische documentatie met aanbevelingen, mogelijke valkuilen en beveiligingstips. Dit maakt het project waardevol voor ontwikkelaars die een gelijkaardig platform willen uitrollen.

Het succes van de graduaatsproef wordt afgemeten aan de volledigheid van de implementatie en de toepasbaarheid van de oplossing in een realistische setting.


% Hier beschrijf je welke resultaten je verwacht. Als je metingen en simulaties uitvoert, kan je hier al mock-ups maken van de grafieken samen met de verwachte conclusies. Benoem zeker al je assen en de onderdelen van de grafiek die je gaat gebruiken. Dit zorgt ervoor dat je concreet weet welk soort data je moet verzamelen en hoe je die moet meten.

% Wat heeft de doelgroep van je onderzoek aan het resultaat? Op welke manier zorgt jouw graduaatsproef voor een meerwaarde?

% Hier beschrijf je wat je verwacht uit je onderzoek, met de motivatie waarom. Het is \textbf{niet} erg indien uit je onderzoek andere resultaten en conclusies vloeien dan dat je hier beschrijft: het is dan juist interessant om te onderzoeken waarom jouw hypothesen niet overeenkomen met de resultaten.



\printbibliography[heading=bibintoc]

\end{document}